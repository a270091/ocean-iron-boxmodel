\documentclass[a4paper,titlepage=false]{scrartcl}

\usepackage{hyperref}
\hypersetup{
    colorlinks=true,
    linkcolor=blue,
    filecolor=magenta,      
    urlcolor=cyan
    }
\usepackage{natbib}

\begin{document}

\title{A simple box model for the geochemical cycling of iron in the ocean}

\author{Christoph V\"olker \\
  Alfred Wegener Institute, \\
  Helmholtz Centre for Polar and Marine Research \\
  Bremerhaven, Germany}
\date{\today}

\maketitle

\section{Overview}

This project contains matlab code to model the distribution of dissolved iron, 
phosphate and of zero, one or two organic iron-binding ligands in the ocean. 
The ocean is discretized into 12 boxes, representing main water masses in the
Atlantic and Indo-Pacific ocean. Boxes are assumed to be homogeneously mixed
and the ocean circulation is prescribed as exchanges between these boxes.

The model has been used to explore feedbacks in the ocean iron cycle
connected to organic iron complexation \citep{Voelker21}. The full
model description can be found there.

The code presented here does not only contain the core model, but
several additional tools. It is grouped broadly into the four parts
described in each of the following sections. The basic code to
reproduce the findings in \citet{Voelker21} is described in sections 2
and 3. This part should run without any modification of the
code. 

Sections 4 and 5 cover additional material that is not needed by the
model, but that may be useful if some used wants to modify the model,
e.g.\ by adding new boxes etc. It may also be interesting to persons
who want to understand how exactly the model boxes were chosen, how
the data was averaged etc. They contain the tools that were used to
construct the boxes, and to average observational data in the model
boxes, plus some that were used to optimize model parameters. The code
described in these two sections will require some action by the user
before it works, mostly to download the data that is used here, and to
set the paths to this data.

\section{Code to integrate the model and to analyse ligand
  feedbacks:}

Several different versions of the box model exist, differing in the
number of modeled quantities in each box (\verb+boxmodel_po4.m+ for
example solves only one quantity, namely the phosphate concentration,
while \verb+boxmodel_po4dopfe2lig_export.m+ solves for five
quantities, namely phosphate, dissolved organic phosphorus, iron, and
two ligands), and in whether export production is calculated from
nutrients (all models ending with \verb+_export.m+), or
prescribed.

Running the models results in two output files, which are
both found in the subdirectory \verb+results+: One ASCII-file
containing the final steady state concentration of the models state
variables in all 12 model boxes in a table (\verb+equil*.dat+), and
one binary matlab file that documents the parameter values used in
that model run (\verb+parameters*.mat+).

Each of these models first calls the routine
\verb+boxmodel_init_params.m+ to initialize all model parameters,
including the volume of the boxes, and the mass transports between
them. The parameters are transferred in the form of a global
structure variable. 

Each of the models then uses a matlab-provided routine to integrate
stiff ordinary equations to integrate the model into equilibrium,
passing the name of the routine that contains the right hand side of
the corresponding box model differential equations to this
routine. These routines have names \verb+boxmodel_dgl_*.m+.

In the equations with prognostic ligands, one important step is to
calculate the 'free', i.e. not organically complexed iron
concentration. In the case of one or two organic ligands this boils
down to finding the roots of either a quadratic or cubic polynomial. I
also experimented with one version where a third weaker ligand was
assumed to be present in constant concentration, as a representation
maybe of humic substances. In that case, I created an extra function
for the calculation of free iron, \verb+func_feprime_tree_ligands.m+. 

Feedbacks between biological production and organic iron complexation
are calculated by running one of the model variants with prognostic
(i.e.\ variable) ligands into equilbrium, then saving the resulting
steady-state ligand concentration, and then running the model again
(under some changed external condition, such as dust iron input), but
this time with ligands held fixed at the previous steady state. This
is done by the routines \verb+feedback_loop_po4dopfelig.m+ and
\verb+feedback_loop_po4dopfe2lig.m+. These routines also need
additional differential equation routines which solve for phosphate,
DOP and iron, but take the ligand concentration as spatially variable
from the previous run with variable ligands. They are called
\verb+boxmodel_dgl_po4dopfe_ligfix_export.m+ and
\verb+boxmodel_dgl_po4dopfe_2ligfix_export.m+.

The result from running the feedback analysis is saved in the
directory \verb+results+ as a matlab binary file
\verb+feedback_*.mat+.

\section{Routines to plot and analyse model output:}

After running the models, or running the feedback analyses, the output
can be plotted using the generated output in the \verb+results+
directory. The plotting routines also print out some basic analyses of
the model output to the screen. Four different plots are provided,
corresponding to the ones in \citet{Voelker21}:

\verb+plot_po4_allmodels_bright.m+ plots modeled phosphate
concentrations in the twelve model boxes against the average phosphate
concentration within the model boxes from the World Ocean Atlas
2009. \verb+plot_po4_allmodels.m+ is an older version with less nice
colours.

\verb+plot_dfe_allmodels_bright.m+ plots modeled dissolved iron
concentration against meadin and quantiles of observed dissolved iron
from the second GEOTRACES intermediate data product.

\verb+plot_ligands_allmodels_bright.m+ plots the steady state ligand
concentrations for all model boxes. Since the database for observed
ligand concentrations is to small for creating reliable median values
for the boxes, no observation data is plotted.

And finally, \verb+plot_feedbacks_allmodels_bright.m+ plots the
variations of a number of important quantities (e.g.\ global average
dissolved iron, and total global export production) as a function of
external dust input, with and without the feedback.

All plotting routines are set up for plotting not only one model run
or one feedback analysis, but comporing several model runs.If you want
to plot just one output of a model run you will have to modify them
slightly.

\section{Code for setup of the model boxes and data sets:}

The parts of the code described here are not necessary if you just
intend to play around with model as it is now, but they may be helpful
if you e.g.\ want to extend the model by adding more boxes, or to
introduce new/additional data sets to compare model output with.

The first step for setting up the model is to calculate the volumes
and, in the case of surface boxes, the surface area of the boxes,
which is done in \verb+calc_boxes_for_model.m+. The same routine also
calculates the average phosphate concentration within the boxes from
World Ocean Atlas 2009 data, and the total export production in each
surface box from the estimate by Schlitzer et al.  A helper function
\verb+read_one_netcdf.m+ is used to read the netcdf files for these
data sets.

To do so, the routine requires several data sets that are not
distributed here with the code, but that should be easily downloadable
from the web. The required files are (names should be clickable links)
\begin{itemize}
\item World Ocean Atlas 2009
  \href{http://data.nodc.noaa.gov/thredds/fileServer/woa/WOA09/NetCDFdata/temperature_annual_1deg.nc}{annual
    average distribution of potential temperature} \citep{Locarnini10}
\item World Ocean Atlas 2009
  \href{http://data.nodc.noaa.gov/thredds/fileServer/woa/WOA09/NetCDFdata/salinity_annual_1deg.nc}{annual
    average distribution of salinity} \citep{Antonov10} 
\item World Ocean Atlas 2009
  \href{http://data.nodc.noaa.gov/thredds/fileServer/woa/WOA09/NetCDFdata/phosphate_annual_1deg.nc}{annual
    average distribution of phosphate} \citep{Garcia10}
\item World Ocean Atlas
  \href{http://iridl.ldeo.columbia.edu/SOURCES/.NOAA/.NODC/.WOA09/.Masks/.basin/data.nc}{basin
    mask} (n.b.: to avoid a confusing filename \verb+data.nc+ that
  file is renamed to \verb+basinmask.nc+, and it is read as such in
  the routine).
\end{itemize}
In addition, the routine uses the seawater equation of state
\verb+swstate.m+ from the WHOI seawater package, which can be downloaded
\href{http://mooring.ucsd.edu/software/matlab/doc/toolbox/ocean/swstate.html}{here},
to calculate potential density from temperature and salinity.

For modelling the iron cycle, external fluxes of iron into the ocean
are required. In the present model, three sources of iron are taken
into account, namely dust deposition, sediment iron release and
hydrothermal input. The fluxes from these sources are calculated in
\verb+integrate_dust_for_boxmodel.m+,
\verb+calc_sediment_for_boxmodel.m+, and
\verb+calc_hydrothermal_for_boxmodel.m+. Again, these routines require
files that are not distributed here with the model:
\begin{itemize}
\item dust deposition is calculated from the deposition field by Luo
  and Mahowald \citep{Mahowald03}; I obtained that field by writing to Natalie
  Mahowald. It should be easy to use other dust deposition files
  instead, e.g.\ from \citet{Albani16}.
\item sediment iron fluxes are calculated from water depth, using the
  \href{https://www.ngdc.noaa.gov/mgg/global/relief/ETOPO2/ETOPO2v2-2006/ETOPO2v2g/netCDF/ETOPO2v2g_f4_netCDF.zip}{ETOPO 2 arc-minutes global topography} \citep{ETOPO2}.
\item hydrothermal iron fluxes are calculated from OCMIP2 \href{http://ocmip5.ipsl.jussieu.fr/OCMIP/phase2/simulations/Helium/boundcond/src_helium.nc}{hydrothermal helium
  fluxes}, as described in \citet{Tagliabue10}
\end{itemize}

Finally, the iron concentrations in the model are compared to values
obtained from the GEOTRACES second intermediate data product bottle
data \citep{Schlitzer18}, which can be downloaded from
\href{https://www.bodc.ac.uk/data/download/asset/2135/generic/}{here}. The
routine to read the GEOTRACES data, sort it into the model boxes, and
calculate quantiles for the individual boxes is
\verb+sort_fe_data_into_boxes.m+.

\section{Routines to optimise parameters of the model:}

\bibliography{modeldescription}
\bibliographystyle{agufull08}



\end{document}

\appendix

\section*{necessary files not contained in this distribution}

Setting up the box model requires that the user has access to a small
set of netCDF files describing distribution of properties within the
ocean. All are freely downloadable from the web, but not distributed
with the model. These files are

World Ocean Atlas 2009: annual average distributions of
- Potential temperature
  (http://data.nodc.noaa.gov/thredds/fileServer/woa/WOA09/NetCDFdata/temperature_annual_1deg.nc)
- Salinity
  (http://data.nodc.noaa.gov/thredds/fileServer/woa/WOA09/NetCDFdata/salinity_annual_1deg.nc)
- Phosphate
  (http://data.nodc.noaa.gov/thredds/fileServer/woa/WOA09/NetCDFdata/phosphate_annual_1deg.nc)
- World Ocean Atlas basinmask
  (http://iridl.ldeo.columbia.edu/SOURCES/.NOAA/.NODC/.WOA09/.Masks/.basin/data.nc)

GEOTRACES second intermediate data product botle data
  (https://www.bodc.ac.uk/data/download/asset/2135/generic/)

Schlitzer global export production



